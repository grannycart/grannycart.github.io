% !TEX TS-program = xelatex
% Above line should force texshop to compile with XeLaTeX - update: seems like these lines are necessary for texshop on Mac, but not texlive on linux
%!TEX encoding =  UTF-16


% :set lbr % Gives you line breaks on words
% :set spell % spell check on
% :set breakindent % wraps lines to same as indent!
% :set autoindent % indents next line same as previous --- though maybe only indents at tab?
% :set foldmethod=indent % folds on indents - only works if indents are tabs though? Also consider foldmethod=manual
% :set foldcolumn=1 % turns on column that shows where folds are
% za = open or close current fold zo = open fold zc = close fold v{motion]zf = create fold manually zd = delete fold zR = open all folds zM = close all folds



% ----------------------------------------------------------------------------------------%
%	Originally Created by Alessandro with TeXShop					%
%	---->	May 27, 2009								%
%	Compiled with XeLaTeX, on Mac OS X						%
%	Licensed under the Creative Commons Attribution 3.0 Unported			%
%	Share, change, spread, and have fun!						%
%	http://creativecommons.org/licenses/by/3.0/					%
%	You can find more at http://aleplasmati.comuv.com				%
% ----------------------------------------------------------------------------------------%


%%%%%%%%%%% Compiling notes %%%%%%%%%%%%
% Compile with xelatex instead of pdflatex
% On mac install basictex. Then use "sudo tlmgr install titlesec bbding" to install necessary extra packages. Download Gillius ADF and install it with Font book to get the necessary open-source Gill sans knock off this requires.
% alternatively, on Mac you could use real Gill sans. See inline notes.
% xelatex needs the package texlive-xetex on Ubuntu. 
% On ubuntu bbding needs the package texlive-fonts-extra
% To install Gillius ADF on ubuntu: find out if you have a font installed, and what to call it with fontspec by using fc-list and grep the output for the font you want
%
% Compile on command line with xelatex [filename].tex


\documentclass[letterpaper]{article}



% See geometry.pdf to learn the layout options. There are lots.
\usepackage[hmargin=0.5in, vmargin=0.5in]{geometry}               
%symbols - the ones you see on the left of the email and of the phone
\usepackage{bbding} 
%Colors/Graphics
\usepackage{color,graphicx}
\usepackage[usenames,dvipsnames]{xcolor}
\setlength{\parindent}{0pt} % set paragraph indent to 0


% NOTE: For some reason normal latex stuff like --- and ``'' don't
% work on this document. Use: –
%Fonts and Tweaks for XeLaTeX
% \defaultfontfeatures{Mapping=tex-text}
% \setmainfont{Futura}
% \setromanfont[Mapping=tex-text]{Hoefler Text} 
% \setsansfont[Scale=MatchLowercase,Mapping=tex-text]{Gill Sans}
% \setmonofont[Scale=MatchLowercase]{Andale Mono}
% setting the fonts was fucking up my sans stuff.  Using \fontspec{} in the places
% where fonts change instead.
\usepackage{fontspec,xltxtra,xunicode}

%\hyphenpenalty=5000
%\tolerance=1000 % these 2 lines minimize hyphenation


%Setup hyperref package, and colours for links, text and headings
% Do a link in the document like this: \href{http://welcometocup.org/Store?product_id=64}{guidebook about zoning}
\usepackage{hyperref}
%\definecolor{linkcolour}{HTML}{FF0080}	%light purple link for the email
\definecolor{linkcolour}{HTML}{0000EE}	%MT: changed html link to blue color
\definecolor{shade}{HTML}{D4D7FE}	%light blue shade
\definecolor{text1}{HTML}{2b2b2b}	%text is almost black
% \definecolor{headings}{HTML}{704214} 	%Sepia, change this if you want to change the headings colors
\definecolor{headings}{HTML}{701112} 	%dark red

\hypersetup{	colorlinks,breaklinks,
			urlcolor=linkcolour, 
			linkcolor=linkcolour}

\usepackage{fancyhdr}				%custom footer
\pagestyle{fancy}
\fancyhf{}
% \rfoot{\color{headings} {\sffamily Last update: \today}. Typeset with X\LaTeX}
%this footer line is useless
\renewcommand{\headrulewidth}{0pt}

\usepackage{titlesec}				%custom \section

\titleformat{\section}
	{\color{headings}
		% \scshape\Large\raggedright}{}{0em}{}[\color{headings}\titlerule] % Section header with horizonal rule
		\scshape\Large\raggedright}{}{0em}{}[\color{headings}] % Section header without horizonal rule

\titlespacing{\section}{0pt}{0pt}{5pt}

\newenvironment{flushRnew}{\bgroup\vspace{0mm}\flushright}{\endflushright\egroup}


\begin{document}
% \fontspec{Gill Sans} % probably works on Mac, but not on ubuntu
\fontspec{Gillius ADF}

\color{text1} % set text color for the whole doc
%<<<< TITLE >>>>

	
\Huge \textbf{Mark Torrey} \hfill \small 26 Hinckley Place \#1, Brooklyn, NY 11218 \color{headings}\raisebox{-3pt}{\Phone} \color{text1}(617) 519 1784 \color{headings}{\raisebox{-3pt}{\Envelope}} \color{text1}mt493@cornell.edu\\
\vspace{14pt} % adjusts space from heading line to content
\color{headings}\hrulefill % give a horizontal rule in heading color
\color{text1} % set text color back to black

% \fontspec{Gill Sans MT} % probably works on Mac, but not on ubuntu
% I guess on Mac I originally used this Microsoft variant on Gill Sans or something for the rest of the document, for some reason.

% \clearpage \vspace*{\fill} \begin{center} \begin{minipage}{\textwidth} \centering{This is some text to be centred vertically.} \end{minipage} \end{center} \vfill % equivalent to \vspace{\fill} \clearpage 
%This stuff centers the content on the page:
\vspace*{\fill}
\begin{center}

	\begin{minipage}[t]{13cm} % Set width of header for middle section here.
		% This should match width of table columns
\large
	\vspace{0pt}	%trick
	
\section{Essay Prompt Response}
\begin{tabular}{p{13cm}}%  p{4.4cm}} 
	% Kinda a hack here, but using a one-column table to make the middle text smaller than the titlesec. Commented out second column width. Syntax for second column (if you need it later for some reason) here:
%	$\bullet$~Community Education	& $\bullet$~Public Policy\\
Equity is not the same as affordability. The Hudson Yards development
here in NYC includes a number of units affordable to very low-income
families. But if a low-income family could afford to live anywhere in
the city, would they choose the shiny new Hudson Yards neighborhood?
I doubt it. It is simply too sterile a place, too manicured, too
\emph{designed} to be a comfortable place for anyone who is not a
billionaire to live. Even if low-income people can afford housing in
a neighborhood, to be an \emph{equitable} neighborhood they need to
feel \emph{comfortable} there too. So how do we create comfortable and
equitable neighborhoods?
~\\ % Carriage return
~\\ % Carriage return
I believe we need something akin to a "hygiene hypothesis" for cities.
You are probably familiar with this notion from medicine – that
as humans have sanitized our environment, our immune systems have
become bored with nothing to do so they start over-reacting, giving us
degenerative and allergic diseases. Similarly, when wealthy people shape
cities, they so sanitize our built environment that our neighborhoods
face the city equivalent of degenerative disease.
~\\ % Carriage return
~\\ % Carriage return
Currently planners, architects, designers, developers, and even
community activists have few tools to address neighborhoods in any way
other than just fighting for who gets access to new things. We need to
work on developing policies that encourage and protect the "lived-in"
feeling of comfortable neighborhoods. When we do permit new things to
be added to a neighborhood, we should consider the way the materials
wear – so even if the building or neighborhood is new now, hopefully
it will feel "lived-in" someday. And we need to continually apply a
critical perspective to make sure that valuing the lived-in part of a
neighborhood is not used as an excuse to neglect neighborhoods that need
investment from the city.

\end{tabular}\\
% ~\\ % I think this just acts as a carriage return outside the table, if you need it
\end{minipage} %END of minipage
\end{center}
\vfill
\clearpage
%
\end{document}  


